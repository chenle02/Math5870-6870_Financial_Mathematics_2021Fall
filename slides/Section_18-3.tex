\def\mySecNum{18.3}
\mySection{\mySecNum~A lognormal model of stock prices}
%-------------- start slide -------------------------------%{{{ 1
\begin{frame}[fragile,t]
	\begin{itemize}
		\item Let $R(t,s)$ be the continuously compounded return from time $t$ to a later time $s$.
		\item For $ t_0<t_1<t_2$, $R(\cdot,\cdot)$ has to satisfy the additivity property:
			\begin{equation*}
				R(t_0,t_2) = R(t_0,t_1) + R(t_1,t_2)
			\end{equation*}
			\bigskip
		\item For time interval $[0,T]$ divided into $n$ subintervals of equal length $T/n$, we have
			\begin{equation*}
				R(0,T) = R(0,h) + R(h,2h) + \cdots + R((n-1)h,T)
			\end{equation*}
		\item[] Assume that
			\begin{equation*}
				\E\left(R((i-1)h,ih)\right)   = \alpha_h \quad \text{and} \quad
				\Var\left(R((i-1)h,ih)\right) = \sigma_h^2
			\end{equation*}
		\item[] Then
			\begin{equation*}
				\E\left(R(0,T)\right)   = n\alpha_h \quad \text{and} \quad
				\Var\left(R(0,T)\right) = n\sigma_h^2
			\end{equation*}
			\mySeparateLine
			\bigskip
		\item By central limit limit theorem, as $n\to\infty$, one can assume that
			\begin{equation*}
				R(0,T)\sim N
			\end{equation*}
	\end{itemize}
\end{frame}
%-------------- end slide -------------------------------%}}}
%-------------- start slide -------------------------------%{{{ 1
\begin{frame}[fragile]
\begin{equation*}
	\ln(S_t / S_0) \sim N\left([\alpha-\delta-0.5\sigma^2]t, \sigma^2 t\right)
\end{equation*}
\bigskip

\begin{equation*}
	\ln(S_t / S_0) = [\alpha-\delta-0.5\sigma^2]t+ \sigma \sqrt{t}\: Z
\end{equation*}
\bigskip

\begin{equation*}
	S_t= S_0 e^{[\alpha-\delta-0.5\sigma^2]t} e^{\sigma \sqrt{t} \: Z}
\end{equation*}
\bigskip

\mySeparateLine
\bigskip

\begin{equation*}
	\E[S_t]                   = S_0 e^{[\alpha-\delta]t} \quad \text{and} \quad
	\text{Median stock price} = e^{[\alpha-\delta-0.5\sigma^2]t}
\end{equation*}

\begin{equation*}
	\text{One standard deviation}
	\begin{cases}
		\text{move up} = e^{[\alpha-\delta-0.5\sigma^2]t \textcolor{magenta}{+} \sigma \sqrt{t} \times 1}  \\[1em]
		\text{move down} = e^{[\alpha-\delta-0.5\sigma^2]t \textcolor{cyan}{-} \sigma \sqrt{t} \times 1}
	\end{cases}
\end{equation*}
\end{frame}
%-------------- end slide -------------------------------%}}}
%-------------- start slide -------------------------------%{{{ 1
\begin{frame}[fragile]
	\begin{center}
		Go over examples 18.4 and 18.5 on textbook on p. 555.
	\end{center}
\end{frame}
%-------------- end slide -------------------------------%}}}
